\documentclass{article}

\usepackage[brazilian]{babel}
\usepackage[utf8]{inputenc}
\usepackage[T1]{fontenc}
\usepackage{tcolorbox}
\usepackage{amsmath}
\usepackage{amsfonts}
\usepackage{lmodern}


% bib management

\usepackage[style = abnt, backend = biber]{biblatex}
\addbibresource{refs.bib}

% This allows us to add our colors
\usepackage{xcolor}

%just input either solarized light or dark to change between the colors
%\input{solarizeddark.tex}
\input{solarizedlight.tex}

\input{styling.tex}


\title{\textcolor{red}{Lista de Leitura sobre Institucionalismo Algorítimico}}
\author{Marcelo Veloso Maciel } \date{}

\begin{document}

% Global Page color base03
\pagecolor{base03}
% Global Text color base1
\color{base1}
\maketitle



\section{Introdução}



Acho interessante pensar essa abordagem em duas perspectivas: como o fronte
teórico da Politologia Computacional e como um reenquadramento da teoria
política formal, superconjunto do institucionalismo da escolha racional. Passar
pelo institucionalismo da escolha racional é necessário, na medida que dele que
vem \(99 \% \) da modelagem em política \cite{austen1998social}. Vou butar
alguns textos aqui que lembro now que talvez possam ser relevantes. 


\section{Ciência Social Computacional}

\begin{itemize}
\item O manifesto da área : \cite{conte2012manifesto}
\item Dar uma sacada no texto de Revilla :
  \textcite{cioffi2014introduction}\footnote{A bib entry ta errada. depois ajeito
    dsehirsier}.
\item Achei relevante uma discussao feita no c1(2?) de
  \textcite{weisberg2012simulation} em que ele defende a distinção entre
  modelagem matemática e computacional.
  \begin{itemize}
  \item Vou ler esse livro todo e jogo minhas notas numa pasta so pra isso
  \end{itemize}
\end{itemize}


\section{Institucionalismo}

\begin{itemize}
\item O clássico que já tais lendo : \textcite{hall1996political}
\item Sacar as entradas na seção sobre institucionalismo em
  \textcite{goodin2009oxford};
\item Ostrom tem vários textos daora sobre institucionalismo:
  \begin{itemize}
  \item \textcite{ostrom1986agenda};
  \item \textcite{crawford1995grammar}; 
  \item A primeira parte de \textcite{ostrom2005understanding};
  \end{itemize}
\end{itemize}

\section{Abordagem Algorítimica}

\subsection{Em geral}

\begin{itemize}
\item Esse e o seguinte sao fundacionais pro
  argumento \textcite{sep-social-procedures} \footnote{Sacar as referências desse
    daqui} ;
\item Topzera \(\rightarrow \) \textcite{eijck2009discourses}; 
\item Tem umas coisinhas legais nesse artigo : \textcite{page2008uncertainty}
\end{itemize}


\subsection{Social Choice}
\begin{itemize}
\item Sacar o texto de Ragan em \textcite{heckelman2015handbook};
\item Obviamente \(\rightarrow\) \textcite{brandt2016handbook};
\end{itemize}

\subsection{GT}
\begin{itemize}
\item Livro de Tardos - ver só o primeiro artigo de conceitualização de Algo GT \textcite{nisan2007algorithmic}
\end{itemize}

\begin{itemize}
\item Aqui deves tar sabendo o que ta rolando. Bota aqui as referencias que tu
  consideras relevante etc. 
\end{itemize}



\printbibliography

\end{document}
