\documentclass{article}

\usepackage[brazilian]{babel}
\usepackage[utf8]{inputenc}
\usepackage[T1]{fontenc}
\usepackage{tcolorbox}
\usepackage{amsmath}
\usepackage{amsfonts}
\usepackage{lmodern}


% bib management

\usepackage[style = abnt, backend = biber]{biblatex}
\addbibresource{refs.bib}

% This allows us to add our colors
\usepackage{xcolor}

%just input either solarized light or dark to change between the colors
%\input{solarizeddark.tex}
% Background Tones
\definecolor{base3}{HTML}{002b36} % for solarized dark
\definecolor{base2}{HTML}{073642} 
\definecolor{base02}{HTML}{eee8d5} % for solarized light
\definecolor{base03}{RGB}{253,246,227}

% Content Tones
\definecolor{base1}{HTML}{586e75}
\definecolor{base0}{HTML}{657b83}
\definecolor{base00}{HTML}{839496}
\definecolor{base01}{HTML}{93a1a1}

% Accent Colors
\definecolor{cyan}{HTML}{2aa198}
\definecolor{violet}{HTML}{6c71c4}
\definecolor{yellow}{HTML}{b58900}
\definecolor{orange}{HTML}{cb4b16}
\definecolor{red}{RGB}{220,50,47}
\definecolor{magenta}{HTML}{d33682}
\definecolor{blue}{HTML}{268bd2}
\definecolor{cyan}{HTML}{2aa198}
\definecolor{green}{HTML}{859900}

\usepackage{tcolorbox}
\tcbset
{every box/.style={colback=base03,colframe=blue,boxrule=2mm,bottom=-7pt}}

% We will use this to alter commands and environments, making equations and tables different colors
\usepackage{etoolbox}
% All equation environments are base2
\AtBeginEnvironment{equation}{\color{base2}}
\AtBeginEnvironment{equation*}{\color{base2}}
\AtBeginEnvironment{math}{\color{violet}}
\AtBeginEnvironment{align}{\color{base2}}
\AtBeginEnvironment{tabular}{\color{violet}}
\usepackage{cancel}
\renewcommand{\CancelColor}{\color{red}}

% A nice sans- serif font
\usepackage[default,osfigures,scale=0.95]{opensans}

% Un-italicized the equations and makes them easier to read. Check out
% http://www.biwako.shiga-u.ac.jp/sensei/kumazawa/tex/newtx.html
% for more options
\usepackage{amsmath}
\usepackage{cmbright}

% Doesn't seem to be working in Share LaTeX
% Gives you control over footer, header, etc. so we can have a highlighter page number
% We also have the section title up on top
\usepackage{fancyhdr}
\pagestyle{fancy}
\cfoot{\textcolor{violet}{\thepage} }
% section title on the left side of the page
\fancyhead[L]{\textcolor{violet}{\slshape \leftmark}}
\fancyhead[R]{}

% Allows us to style section headings
\usepackage{sectsty}
\sectionfont{\color{yellow}}
\subsectionfont{\color{orange}}
\subsubsectionfont{\color{green}}

%Control over figure captions.  I bolded them as well
\usepackage[font={color=cyan,bf}]{caption}



\title{\textcolor{red}{Lista de Leitura sobre Institucionalismo Algorítimico}}
\author{Marcelo Veloso Maciel } \date{}

\begin{document}

% Global Page color base03
\pagecolor{base03}
% Global Text color base1
\color{base1}
\maketitle



\section{Introdução}



Acho interessante pensar essa abordagem em duas perspectivas: como o fronte
teórico da Politologia Computacional e como um reenquadramento da teoria
política formal, superconjunto do institucionalismo da escolha racional. Passar
pelo institucionalismo da escolha racional é necessário, na medida que dele que
vem \(99 \% \) da modelagem em política \cite{austen1998social}. Vou butar
alguns textos aqui que lembro now que talvez possam ser relevantes. 


\section{Ciência Social Computacional}

\begin{itemize}
\item O manifesto da área : \cite{conte2012manifesto}
\item Dar uma sacada no texto de Revilla :
  \textcite{cioffi2014introduction}\footnote{A bib entry ta errada. depois ajeito
    dsehirsier}.
\item Achei relevante uma discussao feita no c1(2?) de
  \textcite{weisberg2012simulation} em que ele defende a distinção entre
  modelagem matemática e computacional.
  \begin{itemize}
  \item Vou ler esse livro todo e jogo minhas notas numa pasta so pra isso
  \end{itemize}
\end{itemize}


\section{Institucionalismo}

\begin{itemize}
\item O clássico que já tais lendo : \textcite{hall1996political}
\item Sacar as entradas na seção sobre institucionalismo em
  \textcite{goodin2009oxford};
\item Ostrom tem vários textos daora sobre institucionalismo:
  \begin{itemize}
  \item \textcite{ostrom1986agenda};
  \item \textcite{crawford1995grammar}; 
  \item A primeira parte de \textcite{ostrom2005understanding};
  \end{itemize}
\end{itemize}

\section{Abordagem Algorítimica}

\subsection{Em geral}

\begin{itemize}
\item Esse e o seguinte sao fundacionais pro
  argumento \textcite{sep-social-procedures} \footnote{Sacar as referências desse
    daqui} ;
\item Topzera \(\rightarrow \) \textcite{eijck2009discourses}; 
\item Tem umas coisinhas legais nesse artigo : \textcite{page2008uncertainty}
\end{itemize}


\subsection{Social Choice}
\begin{itemize}
\item Sacar o texto de Ragan em \textcite{heckelman2015handbook};
\item Obviamente \(\rightarrow\) \textcite{brandt2016handbook};
\end{itemize}

\subsection{GT}


\begin{itemize}
\item Aqui deves tar sabendo o que ta rolando. Bota aqui as referencias que tu
  consideras relevante etc. 
\end{itemize}



\printbibliography

\end{document}
