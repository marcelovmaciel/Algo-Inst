\documentclass{article}

\usepackage[brazilian]{babel}
\usepackage[utf8]{inputenc}
\usepackage[T1]{fontenc}
\usepackage{tcolorbox}
\usepackage{amsmath}
\usepackage{amsfonts}
\usepackage{lmodern}

\title{ Teoria Política Formal}

\author{ Kaique AcabouOSuco Sauron \\ Marcelo SouFoca Javalizeu }
\documentclass{article}

\date{}

\begin{document}
\maketitle

\newpage

\section{ Introdução }

No presente artigo, pretende-se...

\begin{itemize}

\item Revisão Bibliográfica - Tipos de institucionalismo e suas características

	\subitem Velho Institucionalismo
    
	\subitem Novos Institucionalismos: Histórico, Sociológico e Da Escolha Racional

		Hall and Taylor - As três versões daquela praga chata lá

			-> Institucionalismo Histórico
			
				Segundo Hall e Tayor( 1996 ), o Institucionalismo Histórico surge como uma crítica contra o estrutural-funcionalismo e o conflito grupal presentes no estudo da vida política. Porém, ao invés de rejeitarem completamente os elementos supracitados, Hall e Taylor( 1996 ) afirmam que os estudiosos mantiveram os mesmos, mudando, entretanto, o enfoque dado aos tais.
				No caso do conflito grupal, segundo Hall e Taylor( 1996 ), concordava-se que a disputa entre grupos, no que diz respeito ao acesso à recursos, é vital para a compreensão da vida política. Porém, adicionaram a maneira pela qual o conflito entre a comunidade política e as estruturas econômicas era organizado institucionalmente, fator que gerava o privilegiamente de certos interesses.
				Quanto ao estrutural-funcionalismo, de acordo com Hall e Taylor( 1996 ), os institucionalistas históricos mantiveram apenas o estruturalismo, o qual julgavam intrísecos às instituições que organizavam a comunidade política. Afirmavam que o ordenamento institucional era justamente o responsável pelo comportamento coletivo.
			
			-> Institucionalismo Sociológico
			
				Puro creme de lixo
			
			-> Institucionalismo Da Escolha Racional
			
			

		Goodin - Institutional Design e etc

			

\end{itemize}

\section{ Compt Soc Science } - Breve apresentação do que é ciencia social compt

\begin{itemize}

\item Atualização Conceitual

    \subitem A ideia de instituições como algorítmos
    
        \subsubitem Descrição/Caracterização e Diferenciação
        
\item Vantagens

    \subitem Como o conceito/abordagem engloba os anteriores
    \subitem Como o conceito/abordagem expande/adiciona/permite melhorias ou melhor exploração da noção de instituições

\end{itemize}

\section{ Algo inst } - Definição/conceitualização

\begin{itemize}

\item Tools, Frameworks e Models

    \subitem Exemplos( pegar daqueles livros lá que tu passou e outras coisas )

\end{itemize}

\section{ Vantagens } - O que diabos essa nova definição nos permite?

\section{ Conclusão }

Sum up

\newpage

\section*{ Bibliografia }

\end{document}

