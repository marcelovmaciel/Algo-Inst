\documentclass{article}

\usepackage[brazilian]{babel}
\usepackage[utf8]{inputenc}
\usepackage[T1]{fontenc}
\usepackage{tcolorbox}
\usepackage{amsmath}
\usepackage{amsfonts}
\usepackage{lmodern}

\title{ Teoria Política Formal}

\author{ Kaique AcabouOSuco Sauron \\ Marcelo SouFoca Javalizeu }

\date{}

\begin{document}
\maketitle

\newpage

\section{ Introdução }

\begin{itemize}

\item Revisão Bibliográfica

    \subitem Velho Institucionalismo
    
    \subitem Novos Institucionalismos: Histórico, Sociológico e Da Escolha Racional

\end{itemize}

\section{ Proposta }

\begin{itemize}

\item Atualização Conceitual

    \subitem A ideia de instituições como algorítmos
    
        \subsubitem Descrição/Caracterização e Diferenciação
        
\item Vantagens

    \subitem Como o conceito/abordagem engloba os anteriores
    
        Como engloba -> A ideia seria: assumindo a possbilidade de se unir o Inst. histórico com o da escolha racional, apelar para aquele argumento velho de que modelagem computacional conseguiria lidar justamente com os elementos dos dois Inst( Path Dependence + RCT ). Teria que fazzer a seção anterior( Atualização conceitual ) já pensando nisso.
    
    \subitem Como o conceito/abordagem expande/adiciona/permite melhorias ou melhor exploração da noção de instituições
    
        

\end{itemize}

\section{ Exploração( mudar nome depois ) }

\begin{itemize}

\item Tools, Frameworks e Models

    \subitem Exemplos( pegar daqueles livros lá que tu passou e outras coisas )
    
        Apresentar modelos que representem justamente a ideia de Algo Inst.

\end{itemize}

\section{ Conclusão }

Sum up

\newpage

\section*{ Bibliografia }

\end{document}
